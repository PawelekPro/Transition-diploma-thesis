\usepackage[T1]{fontenc}
\usepackage[english,polish]{babel}
\usepackage[utf8]{inputenc}
\usepackage{lmodern}
\selectlanguage{polish}
\usepackage[left=2.0cm, right=2.0cm, top=2.5cm, bottom=2.5cm]{geometry}
\usepackage{latexsym}
\usepackage{graphicx}
\usepackage{amsmath}
\usepackage{enumerate}
\usepackage{bm}
\usepackage{tikz}
\usepackage{float}
\usetikzlibrary{matrix,shapes,arrows,positioning,chains,intersections}
\usepackage{multirow}
\usepackage{bm}
\usepackage{smartdiagram}
\usepackage{textcomp}
%\usetikzlibrary{shapes,arrows}
\usepackage{tikz}
\usetikzlibrary{graphs}
\usepackage{forest}
\usetikzlibrary{automata,positioning} %to chyba nie dziala, mialo zawijac tekst w drzewku
\usepackage{chngcntr} %to i nizej sa od numerowania rysunkow zgodnie z rozdzialem
\counterwithin{figure}{section} 
\usepackage{textcomp} %greckie literki w tekscie
\usepackage{graphicx} %foty
\usepackage{wrapfig} %zawijanie tekstu 
\usepackage{enumerate} %prosta lista
\usepackage[nottoc,numbib]{tocbibind} %biblio w ToC
%\usepackage{tabu} %grubosc lini w tabeli
\usepackage{chngcntr} %numerowanie tabeli zgodnie z numerem rozdzialu (to nizej tez)
\counterwithin{table}{section} %numerowanie tabel zgodnie z numerem rozdzialu
\usepackage[toc,page]{appendix} %anex
\usepackage{adjustbox} %ograniczenie tabelek do wielkosci strony
\usepackage{pdfpages}

\newsavebox{\temp}
\newlength{\tempwidth}
\newlength{\tempheight}

\newcommand{\addpdf}[1]{%
    \sbox{\temp}{\includegraphics{#1}}%
    \setlength{\tempwidth}{\widthof{\usebox{\temp}}}%
    \setlength{\tempheight}{\heightof{\usebox{\temp}}}%

    \ifthenelse{\tempwidth > \tempheight}
        {\includepdf[fitpaper, templatesize={210mm}{297mm}, scale=0.92, landscape]{#1}}
        {\includepdf[fitpaper, templatesize={210mm}{297mm}, scale=0.92]{#1}}%
}

\newcommand{\myTitle}{KAP_laboratorium_3}
\newcommand{\myName}{Paweł Gilewicz}
\newcommand{\myThesisType}{Sprawozdanie}

% apply config to document settings
\title{\myTitle}
\author{\myName}
\usepackage[unicode, hidelinks]{hyperref}
\hypersetup{
	pdftitle={\myTitle},
	pdfauthor = {\myName},
	pdfsubject={\myThesisType},
	pdfproducer={LaTeX},
	pdfcreator={pdflatex},
	colorlinks=false
}

% MATRIX settings
\usepackage{amsmath}
\makeatletter
\renewcommand*\env@matrix[1][\arraystretch]{%
  \edef\arraystretch{#1}%
  \hskip -\arraycolsep
  \let\@ifnextchar\new@ifnextchar
  \array{*\c@MaxMatrixCols c}}
\makeatother
\usepackage{mathtools}
\usepackage{amsmath}

% DOCUMENT settings
% dot after section/subsection/subsubsection number
\usepackage{secdot}
\sectiondot{subsection}
\sectiondot{subsubsection}

\usepackage{tocloft} %kropki w spisie tresci
\usepackage[labelsep=period]{caption}
\usepackage{graphicx}
\usepackage[font=small, % equivalent to 10 pt font
			labelfont=bf, 
			justification=justified, 
			singlelinecheck=false]{caption} 
\usepackage[justification=centering]{subcaption} % two images side by side captions
\usepackage{booktabs} % pretty LaTeX tables
\usepackage{siunitx} % units SI e.g. \SI{10}{\kilogram\per\meter\square}
\usepackage{mathtools} % amsmath, symbols such as brackets, arrows, equation numbering only for referrenced eqs.
%\usepackage[parfill]{parskip} % spacing between paragraphs instead of indent
%\parfillskip 0pt plus 0.75\textwidth % get rid of widows at the end of paragraphs
\frenchspacing % for "Polish" spaces after the sentence
\usepackage{polski} % Polish rules of hyphenation
\usepackage{dashrule} % for dotted lines in declarations page
\usepackage{emptypage} % removes headers on empty pages
\usepackage{colortbl}

% FONT settings
\usepackage[T1]{fontenc}
\usepackage{helvet} % you can change to 'helvet' for Helvetica clone, 'uarial' for Arial clone, 'lmodern' for Latin Modern
\renewcommand{\familydefault}{\sfdefault} % change font to sans serif
\usepackage{amsfonts} % mathematical fonts
\usepackage{inconsolata} % monospaced font in urls and \texttt
%\usepackage{url}
%\urlstyle{same}

% DEBUG
\usepackage{lipsum}
\usepackage{etoolbox} % removes page number in table of contents
\patchcmd{\chapter}{plain}{empty}{}{}
\usepackage{slantsc}

\usepackage{fancyhdr}%numerki do dwustronnego
\fancyhf{}
\renewcommand{\headrulewidth}{0pt}
\fancyfoot[LE,RO]{\thepage}
\pagestyle{fancy}

